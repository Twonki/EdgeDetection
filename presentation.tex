\documentclass[11pt,a4paper,titlepage,oneside,mathserif,serif]{beamer}
\usepackage[latin1]{inputenc}
\usepackage{amsmath}
\usepackage{amsfonts}
\usepackage{amssymb}
\usepackage{graphicx}
\author{Leonhard Applis}
\title{Edge Detection}
\subtitle{}
\institute{TH N�rnberg} % (optional)
\date{05.11.2018}
\subject{AvBildMed}

\usetheme{PaloAlto}
\usecolortheme{beaver}
\AtBeginSection[]
{
	\begin{frame}
	\frametitle{Table of Contents}
	\tableofcontents[currentsection,currentsubsection]
\end{frame}
}

\begin{document}
\frame{\titlepage}
\section{What makes an Edge?}
\begin{frame}
	Picture of Felix vs Edges of Felix
\end{frame}
\subsection{Problems}
\begin{frame}
\frametitle{Problem I: Low Contrast}

\end{frame}
\begin{frame}
\frametitle{Problem II: Low Contrast}

\end{frame}
\begin{frame}
\frametitle{Problem III: Noise}

\end{frame}
\subsection{Definition}
\begin{frame}
	\frametitle{Definition}
	In Image Processing, an edge can be defined as a set of contiguous pixel positions where an abrupt change of intensity, gray- or color-values occur. Edges represent boundaries between objects and background. Sometimes, the edge-pixel-sequence may be broken due to insufficient intensity difference.(Malay K. Pakhira )
\end{frame}	
\section{Basics of gradient-based edgedetection}

\section{Advanced gradient-based edgedetection}

\section{Compass Operators}

\section{Edge Sharpening}

\end{document}